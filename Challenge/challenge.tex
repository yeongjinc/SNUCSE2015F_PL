\documentclass[a4paper]{article}

\RequirePackage{geometry}
\RequirePackage{indentfirst}
\RequirePackage{kotex}
\RequirePackage{graphicx}
\RequirePackage{url}
\RequirePackage{listings}
\RequirePackage{color}
\RequirePackage{amsmath}
\RequirePackage{mathtools}
\RequirePackage{latexsym}

\setlength{\parindent}{10pt}

\geometry{a4paper, margin=3cm}

\lstset{
  backgroundcolor=\color{white},
  basicstyle=\footnotesize\ttfamily,
  breakatwhitespace=false,
  breaklines=true,
  commentstyle=\color{green},
  keywordstyle=\color{blue},
  language=ML,
  tabsize=2
}

\allowdisplaybreaks

\begin{document}
\title{Programming Language(Fall 2015) Challenge}
\author{2009-11841 최영진}
\maketitle


\section*{Challenge 1 증명 완성 I}
\begin{align*}
\emph{expression}\ e \rightarrow\ & n \\
| \ & x \\
| \ & \lambda x. e \\
| \ & e\ e \\
| \ & e\ +\ e
\end{align*}


\subsection*{Progress}
$\vdash e : \tau$ 이고 $e$ 가 값이 아니면 반드시 진행 $e \rightarrow e'$ 한다. (P)\\
\begin{enumerate}
	\item $e\ =\ n$ 인 경우, $e$가 값이므로 P는 참이다.
	\item $e\ =\ x$ 인 경우, 다른 케이스로 치환되므로 다른 케이스가 모두 맞다면 참이다.
	\item $e\ =\ \lambda x.e$ 인 경우, $e$가 값이므로 P는 참이다.
	\item $e\ =\ {e}_{1}\ {e}_{2}$ 인 경우, $\vdash\ {e}_{1}\ {e}_{2}: \tau$ 이므로  타입추론 규칙에 의해 $\vdash\ {e}_{1}: \tau' \rightarrow \tau$ 이고 $\vdash\ {e}_{2}: \tau'$이다. 따라서 귀납 가정에 의해서,
\begin{enumerate}
	\item ${e}_{1}$이 값이 아니면 진행 ${e}_{1} \rightarrow {e}_{1}'$하고, 이는 곧 프로그램 실행 $\rightarrow$의 정의에 의해  $({e}_{1}\ {e}_{2}) \rightarrow ({e}_{1}'\ {e}_{2})$ 과 같다.
	\item 마찬가지로 ${e}_{1}$이 값이고 ${e}_{2}$가 값이 아니라면 진행 ${e}_{2} \rightarrow {e}_{2}'$하고, 이는 곧 프로그램 실행 $\rightarrow$의 정의에 의해 $({e}_{1}\ {e}_{2}) \rightarrow ({e}_{1}\ {e}_{2}')$과 같다. 
	\item ${e}_{1}$과 ${e}_{2}$가 모두 값이라면, $\vdash {e}_{1} : \tau' \rightarrow \tau$ 일 수 있는 값 ${e}_{1}$은 오직 $ \lambda x.e'$의 경우 뿐이다. 따라서 프로그램 실행 $\rightarrow$의 정의에 의해 ${e}_{1}\ {e}_{2}\ =\ (\lambda x.e')\ {e}_{2} \rightarrow \{{e}_{2}/{x}\}\ e'$으로 진행한다.
\end{enumerate}
\item $e\ =\ {e}_{1}\ +\ {e}_{2}$인 경우, 타입의 정의에 따라 $\vdash {e}_{1} : \iota$ 이고, $\vdash {e}_{2} : \iota$이다.
\begin{enumerate}
	\item ${e}_{1}$이 값이 아니면 진행 ${e}_{1} \rightarrow {e}_{1}'$하고, 이는 곧 프로그램 실행 $\rightarrow$의 정의에 의해  ${e}_{1} + {e}_{2} \rightarrow {e}_{1}' + {e}_{2}$ 과 같다.
	\item 마찬가지로 ${e}_{1}$이 값이고 ${e}_{2}$가 값이 아니라면 진행 ${e}_{2} \rightarrow {e}_{2}'$하고, 이는 곧 프로그램 실행 $\rightarrow$의 정의에 의해 ${e}_{1} + {e}_{2} \rightarrow {e}_{1} + {e}_{2}'$과 같다. 
	\item ${e}_{1}$과 ${e}_{2}$가 모두 값이면 즉, $\iota$ 타입을 가지는 값이면, 이 둘을 더한 $n$을 값으로 가지는 $e'$으로 진행한다.
\end{enumerate}
\end{enumerate}


\subsection*{Preservation}
$\vdash e = \tau$이고 $e \rightarrow e'$이면 $\vdash e' : \tau$.
\begin{enumerate}
	\item $e\ =\ n$ 인 경우, $e$가 값이므로 $\rightarrow$로 진행하지 않는다.
	\item $e\ =\ x$ 인 경우, 다른 케이스로 치환되므로 다른 케이스가 모두 맞다면 참이다. 치환이 타입을 보존하는 것은 아래의 "Preservation under substitution"에서 증명한다.
	\item $e\ =\ \lambda x.e_1$ 인 경우, $e$가 값이므로  $\rightarrow$로 진행하지 않는다.
	\item $e\ =\ {e}_{1}\ {e}_{2}$ 인 경우, $\vdash\ {e}_{1}\ {e}_{2}: \tau$ 이므로  타입추론 규칙에 의해 $\vdash\ {e}_{1}: \tau' \rightarrow \tau$ 이고 $\vdash\ {e}_{2}: \tau'$이다. ${e}_{1} {e}_{2} \rightarrow e'$이라면 세 가지 경우밖에 없다. 
	\begin{enumerate}
		\item ${e}_{1} \rightarrow {e}_{1}'$이라서  $({e}_{1}\ {e}_{2}) \rightarrow ({e}_{1}'\ {e}_{2})$ 인 경우. 귀납 가정에 의해 $\vdash {e}_{1}':\tau' \rightarrow \tau$. $\vdash {e}_{2}: \tau'$이므로, 타입추론 규칙에 의해 $\vdash {e}_{1}'{e}_{2}:\tau$. 
		\item ${e}_{1}$은 값이고 ${e}_{2} \rightarrow {e}_{2}'$이라서 $({e}_{1}\ {e}_{2}) \rightarrow ({e}_{1}\ {e}_{2}')$인 경우. $\vdash {e}_{1} : \tau' \rightarrow \tau$이고, 귀납 가정에 의해 $\vdash {e}_{2} : \tau'$이므로, 타입추론 규칙에 의해 $\vdash {e}_{1} {e}_{2}' : \tau$.
		\item ${e}_{1}$과 ${e}_{2}$가 모두 값이라면, $\vdash {e}_{1} : \tau' \rightarrow \tau$인 값 ${e}_{1}$은 타입추론 규칙에 의해 ${e}_{1} = \lambda x.e'$밖에는 없다. 즉 $e_1 e_2 = (\lambda x.e') v$이고, $(\lambda x.e') v \rightarrow \{v/x\}e'$이다. $\vdash \lambda x.e' : \tau' ->\tau$이라면 타입추론 규칙에 의해 $x : \tau' \vdash e' : \tau$이다. $\vdash v:\tau'$이므로, "Preservation under substitution(아래에서 증명)"에 의해 $\vdash \{v/x\}e' : \tau$이다.
	\end{enumerate}
	\item $e\ =\ e_1\ +\ e_2$인 경우, 타입의 정의에 따라 $\vdash e_1 : \iota$ 이고, $\vdash e_2 : \iota$이다.
	\begin{enumerate}
		\item $e_1$이 값이 아니라면 $e_1 \rightarrow e_1'$과 같이 진행하고(Progress), 귀납가정에 의해 $\vdash e_1' : \iota$이다. 따라서 타입추론 규칙에 의해 $\vdash e_1' + e_2 : \tau$이다.
		\item $e_1$이 값이고 $e_2$가 값이 아니라면 ${e}_{2} \rightarrow {e}_{2}'$과 같이 진행하고(Progress), 귀납가정에 의해 $\vdash e_2' : \iota$이다. 따라서 타입추론 규칙에 의해 $\vdash e_1 + e_2' : \tau$이다.
		\item ${e}_{1}$과 ${e}_{2}$가 모두 값이면 즉, $\iota$ 타입을 가지는 값이면, 이 둘을 더한 $n$을 값으로 가지는 $e'$으로 진행한다. $e = e_1 + e_2$에서 $\tau$는 $\iota$이고, 이 $n$는 $\iota$ 타입이므로 $\vdash e' : \tau$이다.
	\end{enumerate}
\end{enumerate}

\subsection*{Preservation under Substitution}
$\Gamma \vdash v = \tau'$이고 $\Gamma + x : \tau' \vdash e : \tau$이면 $\Gamma \vdash \{v/x\}e : \tau$.
\begin{enumerate}
	\item $e\ =\ n$ 인 경우, 치환할 것이 없으므로 성립한다.
	\item $e\ =\ x$ 인 경우, $x$가 다른 것으로 치환되는데, 귀납 가정에 의해 치환은 타입을 보존하므로 다른 케이스들로 회귀한다. 따라서 다른 케이스들이 모두 맞다면 성립한다.
	\item $e\ =\ \lambda x.e$ 인 경우, 항상 $y \notin \{x\} \cup FVv$인 $\lambda y.e'$로 간주할 수 있으므로 $\{v/x\}\lambda y.e' = \lambda y.\{v/x\}e'.$ 따라서, 보일 것은 $\Gamma \vdash \lambda y.\{v/x\}e' : \tau$. (단 $\tau$는 $\tau_1 \rightarrow \tau_2$로 설정) \\
	가정 $\Gamma + x : \tau' \vdash \lambda y.e' : \tau_1 \rightarrow \tau_2 $으로부터 타입추론 규칙에 의해 $\Gamma + x : \tau' + y : \tau_1 \vdash e' : \tau_2 $이고, $\Gamma \vdash v : \tau'$와 $y \notin FV(v)$ 으로부터 $\Gamma + y : \tau_1 \vdash v  \tau' $ 이므로, 귀납 가정에 의해 $\Gamma + y : \tau_1 \vdash \{v/x\}e' : \tau_2$. 즉, 타입추론 규칙에 의해 $\Gamma \vdash \lambda y.\{v/x\}e' : \tau_1 \rightarrow \tau_2$.
	
	\item $e\ =\ {e}_{1}\ {e}_{2}$ 인 경우, $\vdash\ e_1 : \tau_1$ 및 $\vdash e_2 : \tau_2$라고 하자. $\{v/x\}e = \{v/x\}e_1 \{v/x\}e_2$이고 귀납 가정에 의해 $\vdash \{v/x\}e_1 : \tau_1$ 이고 $\vdash \{v/x\}e_2 : \tau_2$ 이다. 따라서 타입추론 규칙에 의해 $\Gamma \vdash \{v/x\}e_1 \{v/x\}e_2 : \tau_1 \tau_2$ 이고, 이는 보존된 타입 $\tau$이다.
	
	\item $e\ =\ e_1\ +\ e_2$인 경우, 위와 같이 $\vdash\ e_1 : \tau_1$ 및 $\vdash e_2 : \tau_2$라고 하자. $\{v/x\}e = \{v/x\}e_1 + \{v/x\}e_2$이고 귀납 가정에 의해 $\vdash \{v/x\}e_1 : \tau_1$ 이고 $\vdash \{v/x\}e_2 : \tau_2$ 이다. 따라서 타입추론 규칙에 의해 $\Gamma \vdash \{v/x\}e_1 + \{v/x\}e_2 : \tau$이다.
\end{enumerate}


\section*{Challenge 4 재귀호출의 비용}
\subsection*{1. 끝재귀호출(Tail-recursive call)}
Tail-recursive call의 정의는 재귀호출 후 할 일이 아무것도 없다는 뜻이다. 그렇다면 함수 호출 시점의 E는 더 이상 쓸 일이 없을 것이고, 원래 K에 저장하던 E를 저장할 필요가 없다. 함수의 E'로 교체하면서, 기존의 C는 (예전처럼 (C, E)로 저장할 필요가 없으므로) 함수의 C'에 붙이면 된다. 다만 이 모든 동작은 Tail-recursive function에서만 가능하다. 
\subsection*{2. 첨가한 명령어}
Tail-recursive call을 뜻하는 명령어 trcall 을 다음과 같이 정의한다. 다른 모든 부분은 기존 SM5와 같다.\\
\indent $(l::v::(x,C',E')::S, M, E, trcall::C, K) = (S, M\{l \leftarrow v\}, (x,l)::E', C' @ C, K)$\\
\subsection*{3. 새로운 trans함수의 정의}
어떤 함수가 Tail-recursive call인지 판단하려면 다음 두 가지 경우를 조사하면 된다.\\
1) 변환한 함수가 call로 끝나는 경우\\
2) 변환한 함수가 jtr로 끝나고, 분기 대상 중 적어도 하나가 call로 끝나는 경우\\
따라서 변환한 후 위와 같은 경우 call을 trcall로 바꿔주면 된다. trans의 LETF는 다음과 같이 바뀐다. (숙제 5번 모범답안 코드 참조)
\begin{verbatim}
| K.LETF (f, x, e1, e2) ->
(* trans function body *)
let tranced = trans e1 in
let last_cmd = List.hd (List.rev tranced) in
(* check the function is tl *)
let is_tl =
let rec is_last_call cmd =
match cmd with
| K.CALL -> true
| K.JTR(c1, c2) -> is_last_call(c1) || is_last_call(c2)
| _ -> false
in
is_last_call last_cmd in
(* if not tl, then just use CALL, else replace CALL with TRCALL *)
let body =
let rec replace cmdlist =
match cmdlist with
| [] -> []
| hd::tl ->
(match hd with
| K.CALL -> K.TRCALL :: (replace tl)
| _ -> hd :: (replace tl)
)
in
if not is_tl
then tranced
else replace tranced
in
(* then put it to legacy logic *)
let proc_body = (Sm5.BIND f :: body) @ [Sm5.UNBIND; Sm5.POP] in
[Sm5.PUSH (Sm5.Fn (x, proc_body)); Sm5.BIND f] @
trans e2 @
[Sm5.UNBIND; Sm5.POP]
\end{verbatim}

\section*{Challenge 6 더 좋은 let-다형 타입 시스템}
let-다형 타입 시스템의 문제는, 커버할 수 있는 범위가 기대하는 것보다 작다는 것이다. 다음 프로그램을 보자. 
\begin{align*}
&\texttt{let}\ f\ =\ \texttt{malloc}\ (\lambda x. x)\\
&\texttt{in} \\
&\hphantom{in} f\ \texttt{:=}\ (\lambda x. x + 1)\ \texttt{;} \\
&\hphantom{in} (\texttt{!}f)\ \texttt{true}\\
\end{align*}

in 안에서 let에서 정의된 주소에 대입을 할 경우, 애초에 정의된 타입과 달라져서 런타임에 타입 에러가 발생할 수 있다. 이 문제를 해결하기 위해 let-다형 타입 시스템에서는 expansive 라는 추가 검증 방식을 도입하였다. 그러나 expansive는 APP이나 MALLOC인 경우 무조건 true를 리턴하기 때문에 안전(sound)하지만 불완전(uncomplete)한 부분이 크다.

따라서 다음과 같은 타입 시스템을 제안한다. \\
숙제 8 "람다 예보"의 연립방정식 풀이를 통해 어떤 함수호출식에서 호출될 수 있는 함수들의 집합을 구할 수 있다. 이 방식으로 APP에 들어올 수 있는 함수들을 구한 뒤, 그 함수가 generalize하기 안전(sound)하다면 정상적으로 타입 체킹을 하면 된다.
이와 같은 방식으로 다음 프로그램을 타입 체크할 수 있다.
\begin{align*}
&\texttt{let}\ f\ =\ (\lambda y.y)\ (\lambda x. x)\\
&\texttt{in} \\
&\hphantom{in} (\texttt{!}f)\ \texttt{true}\ \texttt{;}\\
&\hphantom{in} (\texttt{!}f)\ \texttt{1}
\end{align*}

위 프로그램은 $let\ E1\ in\ E2$의 $E1$부분에 $(e_1 e_2)$ 형태의 APP expression이 있기 때문에 기존의 타입 시스템에서는 타입 체킹이 불가능하다. 그러나 실행은 잘 되는 프로그램이다. 람다 예보를 통해 오직 Identity function만 통과시키도록 하여도 위와 같은 프로그램은 처리할 수 있다.


\section*{Challenge 9 메모리는 설탕}
\subsection*{참고자료}
15년도 챌린지가 나오기 전에 14년도 챌린지 (http://ropas.snu.ac.kr/~kwang/4190.310/14/challenge.pdf)를 봤는데, 14년도의 이 문제에는 다음 3번 질문, 답변이 있었습니다.\\
\\
3. 그리고 나면, ref e, e := e, !e는 어떻게 표현?\\
프로그램 식 e가 변환된 것을 $\underline{e}$로 표현하면, 다음과 같이 메모리 설탕
을 녹여낼 수 있을 것이다:\\
\begin{align*}
\underline{\textrm{ref}\ e} = \lambda(v, (c,S)).\textrm{let}\ (v',(c',S'))=\underline{e}(v,(c,S))\ \textrm{in}\ (c',(c'+1, (c',v')::S')) \\
\end{align*}
그런데 15년도 챌린지 문서에는 이 내용이 빠진 것을 봤습니다. 이미 읽어본 터라 3번 질의응답을 아는 상태로 문제를 풀었는데, 혹시 문제가 될까봐 말씀드립니다.

\subsection*{메모리 반응식 풀이}
CPS 과제와 마찬가지로 e가 변환된 것을 $\underline{e}$로 표현하였으므로, 메모리 반응식 뿐 아니라 언어의 모든 expression에 대해 $\underline{e}$를 귀납적으로 정의해야 한다. $\underline{e}$는 이전 식의 값과 메모리 표현(counter * list of (주소 * 값))을 받아 새로운 값과 메모리 표현을 내놓는 식이다.

\begin{eqnarray*}
	\underline{x} & = & \lambda(v, (c,S)).(x, (c,S)) \\
	\underline{\lambda x.e} & = & \lambda(v, (c,S)). \\
	& & (\lambda x. \lambda y.(\underline{e}\ y), (c, S)) \\
	\underline{e\ e'} & = & \lambda(v, (c,S)). \\
	& & \textrm{let}\ (v',(c',S')) = \underline{e}(v,(c,S))\ \textrm{in} \\
	& & \textrm{let}\ (v'',(c'',S'')) = \underline{e'}(v',(c',S'))\ \textrm{in} \\
	& & (v'\ v''\ (v'',(c'',S''))) \\
	\underline{\textrm{ref}\ e} & = & \lambda(v, (c,S)).\textrm{let}\ (v',(c',S'))=\underline{e}(v,(c,S))\ \textrm{in}\ (c',(c'+1, (c',v')::S')) \\
	\underline{e := e'} & = & \lambda(v,(c,S)). \\
	& & \textrm{let}\ (v',(c',S')) = \underline{e}(v,(c,S))\ \textrm{in} \\
	& & \textrm{let}\ (v'',(c'',S'')) = \underline{e'}(v',(c',S'))\ \textrm{in} \\
	& & (v'', (c'' + 1,(v',v'')::S'')) \\
	\underline{!e} & = & \lambda(v,(c,S)). \\
	& & \textrm{let}\ (v',(c',S'))=\underline{e}(v,(c,S))\ \textrm{in} \\
	& & \textrm{let rec}\ f\ x\ S =\ \textrm{match S with} \\ 
	& &| (l, v)::S' -> \textrm{if } x = l\ \textrm{then}\ v\ \textrm{else}\ (f\ x\ S') \\
	& &| \textrm{[]} -> \textrm{unbound error} \\
	& &\textrm{in} \\	
	& & (f\ v'\ S', (c', S')) \\
\end{eqnarray*}

위와 같이 정의함으로써 메모리 반응식(ref, :=, !e)을 다른 식의 조합으로 표현할 수 있다. !e의 표현에는 ML의 문법인 match with 및 리스트 관련 값들을 사용할 수 있다고 가정하였는데, 이 역시 설탕으로 기본 식만으로 쓸 수 있을 것이다.

\end{document}